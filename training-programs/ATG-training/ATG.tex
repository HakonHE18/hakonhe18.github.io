\documentclass[11pt]{article}
\title{\Huge ATG Program \\ \large 3x pr. uge}
\date{}
\usepackage{tcolorbox}
\usepackage{enumitem}
\begin{document}
\maketitle
\begin{tcolorbox}[title=\begin{center}Generelle noter\end{center}]
\begin{itemize}
\item \textbf{Progression} \\ Formålet med træning er altid gradvis progression, hvorvidt det er i form af mere vægt, flere reps eller bedre eksevering er ikke vigtigt, men vigtigt er at have en struktur for progression. \\
Hvis programmet f.eks siger \textit{5 sæt á 10 reps} vælges en vægt så alle sæt kan udføres 10 gange \textbf{\underline{og ikke mere}}. \\ Hvis det lykkedes at få 10 reps i alle 5 sæt øges vægten næste træning. \\ Hvis det ikke lykkedes at få 10 reps i alle 5 sæt: \\
Sænk vægten næste træning, hvis det sidste sæt var mere end 3 gentagelser fra målsætningen. \\
Behold samme vægt næste træning, hvis det sidste sæt var indenfor 3 gentagelser af målsætningen. \\
Vigtigst af alt: Hold teknik ens, så der sker progressiv belastning og ikke progressivt snyd.
\end{itemize}
\end{tcolorbox}
\section*{Programmet}
\subsection*{Dag 1}
\begin{center}
\begin{tabular}{|c|c|}
\hline
Øvelse & Sæt $\times$ Reps\\
\hline
ATG Split Squat & $5\times 10$\\
Deficit push up & $5\times10$\\
Jefferson curl & $5\times10$ \\
Tibialis raise & $5\times10$ \\
Enkelt arm row med håndvægt & $5\times10$\\
Side planke & $4\times$\textit{Max} \\
Dumbbell curl skiftevis (Leroy Colbert style) &  $3\times10$ \\
Triceps extension med håndvægt & $4\times 10$ \\
\hline
\end{tabular}
\end{center}
\subsection*{Dag 2}
\subsection*{Dag 3}
\newpage
\section*{Mål}
\begin{itemize}
\item \textbf{ATG Split squat} \\ $25\%$ af kropsvægt for 5 sæt á 5 reps .
\item \textbf{Jefferson curl} \\ 10 reps med $25\%$ af kropsvægt.
\item \textbf{Tibialis raise} \\ 5 sæt á 5 reps med $25\%$ af kropsvægt.
\item \textbf{Single-leg calf raise} \\ 10 reps med $25\%$ af kropsvægt.
\item \textbf{Nordic curl} \\ 10 negative reps.
\end{itemize}
\end{document}
